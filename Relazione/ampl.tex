\documentclass[main.tex]{subfiles}

\lstdefinelanguage{AMPL}{keywords={set,param,var,arc,integer,minimize,maximize,subject,to,node,sum,in,Current,complements,integer,solve_result_num,IN,contains,less,suffix,INOUT,default,logical,sum,Infinity,dimen,max,symbolic
,Initial,div,min,table,LOCAL,else,option,then,OUT,environ,setof,union,all,exists,shell_exitcodeuntil,binary,forall,solve_exitcodewhile,by,if,solve_messagewithin,check,in,solve_result,printf,model,data,solve,for,reset
},sensitive=true,comment=[l]{\#}}

\lstset{frame=tb,
  language=AMPL,
  aboveskip=3mm,
  belowskip=3mm,
  showstringspaces=false,
  columns=flexible,
  basicstyle={\ttfamily},
  numbers=none,
  numberstyle=\tiny\color{gray},
  keywordstyle=\bfseries,
  commentstyle=\textit,
  stringstyle=\color{mauve},
  breaklines=true,
  breakatwhitespace=true,
  tabsize=4
}



\begin{document}

\section{AMPL}

\subsection{Launcher}
Nel seguente file vi sono le istruzioni per poter risolvere il problema presentato nella precedente sezione.
È stato scelto, una volta risolto il problema, di utilizzare i costrutti del linguaggio AMPL per visualizzare
le soluzioni in un formato ``più umano" della semplice visualizzazione del contenuto delle variabili
(tramite il comando \textbf{display}), poiché queste sono in gran parte costituite da valori nulli (in particolare la variabile $x$).
\lstinputlisting[language=AMPL]{../AMPL/Launcher.run}

\subsection{Modello}
Nel seguente file invece è presente la traduzione nel linguaggio AMPL del modello di programmazione lineare presentato nella sezione §2.
\lstinputlisting[language=AMPL]{../AMPL/Modello.mod}

\end{document}