\documentclass[main.tex]{subfiles}
\begin{document}

\section{Descrizione del problema}
\subsection{Il problema}
Il sito web ``Viaggiare LeggeRO" si occupa di analizzare i prezzi delle compagnie di trasporto (in particolar modo ferroviarie e di autobus) per fornire
ai visitatori la miglior scelta possibile per pianificare uno o più viaggi, risparmiando tempo e denaro.\\ \\
Gli studenti fuori sede sono il suo principale target, in quanto essi vogliono cercare di spendere il meno possibile evitando al contempo di passare troppo tempo in viaggio.
Per spendere meno sono consapevoli che è bene effettuare una buona pianificazione in anticipo, acquistando più viaggi (potenzialmente tutti quelli di cui hanno bisogno in un determinato periodo).
A questo proposito, il sito web fornisce la possibilità di pianificare uno o più viaggi, scegliendo fra le corse e le compagnie a disposizione, 
rientrando nel budget richiesto e al tempo stesso minimizzando la durata complessiva del tempo trascorso in viaggio.\\
\textbf{N.B.} In questo problema, per \textit{viaggio} si intende la coppia \textit{tratta di andata} - \textit{tratta di ritorno}.\\ \\
Ai visitatori che accedono al sito vengono richiesti:
\begin{itemize}
    \item la città di partenza, per l'andata, che al ritorno corrisponde alla città di arrivo;
    \item la città di arrivo, per l'andata, che al ritorno corrisponde alla città di partenza;
    \item il numero di viaggi che si vogliono pianificare;
    \item il budget massimo per tutti i viaggi (in euro);
    \item la durata massima per la singola tratta (in ore);
    \item il massimo numero di tratte per ogni compagnia che si intendono effettuare, comprensivo di tratte di andata e di ritorno.
\end{itemize}
L'ultimo caso è reso disponibile per permettere di limitare le tratte da effettuare con una compagnia, ad esempio per scartarla completamente (quando si segna 0), 
oppure per evitare di effettuare troppi viaggi con essa, che potrebbe essere non troppo gradita.\\
Per ogni compagnia di trasporti che lo permette, il sito web è in possesso dei seguenti dati:
\begin{itemize}
    \item il prezzo di ogni tratta disponibile (in euro) da una città di partenza a una di arrivo;
    \item la durata di ogni tratta disponibile (in ore) da una città di partenza a una di arrivo;
    \item l'emissione di buoni sconto per incentivare l'acquisto dei propri biglietti: se svolta, si conosce il loro valore (in euro) e dopo quanti biglietti acquistati vengono emessi.
\end{itemize}
Ovviamente, una compagnia che emette buoni sconto lo fa per incentivare l'acquisto dei propri biglietti, e non quelli delle altre compagnie, 
conseguentemente un buono sconto emesso dalla compagnia \textit{A} non può essere usato per acquistare biglietti della compagnia \textit{B}.\\
\textbf{N.B.} In questo problema, per semplicità, attraverso il sito web ogni compagnia può emettere solo un buono sconto per pianificazione.\\ \\
Il sito dispone attualmente dei dati di quattro compagnie:
\begin{itemize}
    \item Trenia;
    \item FerrovieItaliane;
    \item BusTravel;
    \item FastBus.
\end{itemize}
L'emissione dei buoni sconto è regolata per ogni compagnia secondo le seguenti politiche:
{
\renewcommand{\arraystretch}{2}
\begin{longtable}[h]{c | c | c}
\textbf{Compagnia di trasporto} & \textbf{Valore} & \textbf{Biglietti richiesti per l'emissione} \\
\hline
\endhead
Trenia                          & \e{15.00}       & 5                                            \\
\hline
FerrovieItaliane                & \e{10.00}       & 4                                            \\
\hline
BusTravel                       & \e{10.00}       & 0 $^{*}$                                     \\
\hline
FastBus                         & -               & - $^{**}$                                    \\
\end{longtable}
}
\setlength{\parindent}{0em}
$^*$ La compagnia BusTravel è appena entrata nel mercato italiano e per autopromuoversi offre un buono sconto fin dal primo acquisto (quindi acquistando il primo biglietto si può già usare il buono sconto).\\
$^{**}$ La compagnia FastBus non offre buoni sconto.

\subsection{Dati di esempio}
Viene fornito un esempio di ricerca effettuata da un utente del sito web:
\begin{itemize}
    \item \textbf{città di partenza}: Padova;
    \item \textbf{città di arrivo}: Torino;
    \item \textbf{numero di viaggi}: 6;
    \item \textbf{budget massimo per tutti i viaggi}: 200 euro;
    \item \textbf{durata massima per la singola tratta}: 5 ore;
    \item \textbf{massimo numero di tratte per ogni compagnia}:
    {
    \renewcommand{\arraystretch}{2}
    \begin{longtable}[h]{c | c}
    \textbf{Compagnia di trasporto} & \textbf{Tratte max.} \\
    \hline
    \endhead
    Trenia                          & 3                    \\
    \hline
    FerrovieItaliane                & 3                    \\
    \hline
    BusTravel                       & 3                    \\
    \hline
    FastBus                         & 3                    \\
    \end{longtable}
    }
\end{itemize}
Vengono di seguito presentati i dati a disposizione del sito per le tratte offerte dalle compagnie disponibili per il tragitto Padova $\leftrightarrow$ Torino (Padova $\rightarrow$ Torino e Torino $\rightarrow$ Padova).\\
\textbf{N.B.} Le celle contenenti ``n.d." indicano un'assenza della corsa.

\subsubsection*{Padova $\rightarrow$ Torino: costi della tratta (in euro)}
{
\renewcommand{\arraystretch}{2}
\begin{longtable}[h]{c | c | c | c | c | c | c | c}
\textbf{Compagnia di trasporto} & \textbf{1} & \textbf{2} & \textbf{3} & \textbf{4} & \textbf{5} & \textbf{6} & \textbf{7} \\
\hline
\endhead
Trenia                          & \e{22.90}  & \e{37.90}  & \e{22.90}  & \e{45.90}  & \e{22.90}  & n.d.       & n.d.       \\
\hline
FerrovieItaliane                & \e{26.90}  & \e{45.80}  & \e{26.90}  & \e{31.90}  & \e{29.90}  & \e{21.90}  & \e{19.90}  \\
\hline
BusTravel                       & \e{11.99}  & \e{15.99}  & \e{13.99}  & n.d.       & n.d.       & n.d.       & n.d.       \\
\hline
FastBus                         & \e{13.90}  & \e{16.90}  & \e{12.90}  & \e{10.90}  & n.d.       & n.d.       & n.d.       \\
\end{longtable}
}

\subsubsection*{Padova $\rightarrow$ Torino: tempi di percorrenza della tratta (in ore)}
{
\renewcommand{\arraystretch}{2}
\begin{longtable}[h]{c | c | c | c | c | c | c | c}
\textbf{Compagnia di trasporto} & \textbf{1} & \textbf{2} & \textbf{3} & \textbf{4} & \textbf{5} & \textbf{6} & \textbf{7} \\
\hline
\endhead
Trenia                          & 4 h        & 4 h        & 4.5 h      & 4 h        & 4 h        & n.d.      & n.d.        \\
\hline
FerrovieItaliane                & 3 h        & 3.25 h     & 3.5 h      & 4 h        & 3.75 h     & 3.5 h     & 3.5 h       \\
\hline
BusTravel                       & 5 h        & 5.3 h      & 4.8 h      & n.d.       & n.d.       & n.d.      & n.d.        \\
\hline
FastBus                         & 5.1 h      & 5 h        & 5.3 h      & 5 h        & n.d.       & n.d.      & n.d.        \\
\end{longtable}
}

\subsubsection*{Torino $\rightarrow$ Padova: costi della tratta (in euro)}
{
\renewcommand{\arraystretch}{2}
\begin{longtable}[h]{c | c | c | c | c | c | c | c}
\textbf{Compagnia di trasporto} & \textbf{1} & \textbf{2} & \textbf{3} & \textbf{4} & \textbf{5} & \textbf{6} & \textbf{7} \\
\hline
\endhead
Trenia                          & \e{34.90}  & \e{37.90}  & \e{32.90}  & \e{17.90}  & \e{22.90}  & \e{17.90}  & \e{16.90}  \\
\hline
FerrovieItaliane                & \e{25.90}  & \e{23.90}  & \e{26.90}  & \e{20.90}  & \e{19.90}  & n.d.       & n.d.       \\
\hline
BusTravel                       & \e{15.99}  & \e{15.99}  & \e{11.99}  & \e{13.99}  & \e{11.99}  & \e{10.99}  & n.d.       \\
\hline
FastBus                         & \e{11.90}  & \e{12.90}  & \e{9.90}   & \e{15.90}  & \e{13.90}  & n.d.       & n.d.       \\
\end{longtable}
}

\subsubsection*{Torino $\rightarrow$ Padova: tempi di percorrenza della tratta (in ore)}
{
\renewcommand{\arraystretch}{2}
\begin{longtable}[h]{c | c | c | c | c | c | c | c}
\textbf{Compagnia di trasporto} & \textbf{1} & \textbf{2} & \textbf{3} & \textbf{4} & \textbf{5} & \textbf{6} & \textbf{7} \\
\hline
\endhead
Trenia                          & 4 h        & 4 h        & 4 h        & 4 h        & 4.5 h      & 4 h        & 4.5 h      \\
\hline
FerrovieItaliane                & 3.6 h      & 3.25 h     & 3 h        & 4 h        & 3.5 h      & n.d.       & n.d.       \\
\hline
BusTravel                       & 5.5 h      & 4.6 h      & 5.5 h      & 5 h        & 5.6 h      & 6 h        & n.d.       \\
\hline
FastBus                         & 5 h        & 5.5 h      & 4.6 h      & 6 h        & 5 h        & n.d.       & n.d.       \\
\end{longtable}
}

\end{document}