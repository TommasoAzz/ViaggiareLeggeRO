\documentclass[main.tex]{subfiles}
\begin{document}

\section{Esecuzione del problema}
\subsection{File dati}
Per provare il modello vengono forniti sei file \textbf{.dat}, ognuno contenente i dati a disposizione del sito al momento della richiesta, oltre alle richieste
dell'utente indicate in §1.1:
\begin{itemize}
    \item Dati\_PdTo\_0.dat, per il tragitto Padova $\leftrightarrow$ Torino (corrispondente ai dati presentati in §1.2);
    \item Dati\_PdTo\_1.dat, per il tragitto Padova $\leftrightarrow$ Torino;
    \item Dati\_MiRm\_0.dat, per il tragitto Milano $\leftrightarrow$ Roma;
    \item Dati\_MiRm\_1.dat, per il tragitto Milano $\leftrightarrow$ Roma;
    \item Dati\_BoBz\_0.dat, per il tragitto Bologna $\leftrightarrow$ Bolzano;
    \item Dati\_BoBz\_1.dat, per il tragitto Bologna $\leftrightarrow$ Bolzano.
\end{itemize}
Più precisamente, le coppie di file:
\begin{itemize}
    \item Dati\_PdTo\_0.dat e Dati\_PdTo\_1.dat,
    \item Dati\_MiRm\_0.dat e Dati\_MiRm\_1.dat,
    \item Dati\_BoBz\_0.dat e Dati\_BoBz\_1.dat,
\end{itemize}
presentano gli stessi dati, ad eccezione del parametro $maxTratte \in \mathbb{Z}_+^{|T|}$ che nel secondo di ciascuna coppia di file viene istanziato 
con dati che lo rendono di fatto ridondante (poiché viene istanziato con il massimo numero di tratte che deve essere necessariamente effettuato).\\
Il motivo di questa scelta è dovuto al fatto di voler analizzare quanto varia la soluzione del modello rendendo meno stringenti i vincoli.\\
Per il tragitto Bologna $\leftrightarrow$ Bolzano, il sito web ha a disposizione i dati della compagnia DeutscherZug oltre a quelli delle compagnie già presentate in precedenza. 
Questa compagnia offre un buono sconto di \e{5} fin dal primo acquisto, come la compagnia BusTravel.

\subsection{Risultati dell'esecuzione}
\subsection*{Tragitto: Padova $\leftrightarrow$ Torino}
Viene richiesta una pianificazione con questi dati:
\begin{itemize}
    \item Budget (massimo): \e{200};
    \item Durata (massima di una tratta): 5h;
    \item Numero viaggi da pianificare: 6;
    \item con Dati\_PdTo\_0.dat: $maxTratte^T = [3,3,3,3]$;
    \item con Dati\_PdTo\_1.dat: $maxTratte^T = [12,12,12,12]$.
\end{itemize}
Il risultato ottenuto è il seguente:
\begin{itemize}
    \item totale di ore in viaggio minime per la pianificazione richiesta:
    \begin{itemize}
        \item con Dati\_PdTo\_0.dat: 49.20h
        \item con Dati\_PdTo\_1.dat: 45.00h
    \end{itemize}
    \item corse selezionate (mostrate solo quelle con $x_{t,c,v} > 0$):
    \begin{itemize}
        \item con Dati\_PdTo\_0.dat
        \begin{itemize}
            \item $x_{Trenia,4,Ritorno} = 3$
            \item $x_{FerrovieItaliane,1,Andata} = 3$
            \item $x_{BusTravel,3,Andata} = 3$
            \item $x_{FastBus,3,Ritorno} = 3$
        \end{itemize}
        \item con Dati\_PdTo\_1.dat
        \begin{itemize}
            \item $x_{FerrovieItaliane,7,Andata} = 5$
            \item $x_{FerrovieItaliane,3,Ritorno} = 1$
            \item $x_{FerrovieItaliane,5,Ritorno} = 3$
            \item $x_{BusTravel,3,Andata} = 1$
            \item $x_{FastBus,3,Ritorno} = 2$
        \end{itemize}
    \end{itemize}
    \item budget impiegato per ogni compagnia (mostrati solo quelli con $b_t > 0$):
    \begin{itemize}
        \item con Dati\_PdTo\_0.dat
        \begin{itemize}
            \item $b_{Trenia} = 53.70$ euro
            \item $b_{FerrovieItaliane} = 80.70$ euro
            \item $b_{BusTravel} = 31.97$ euro (usando la gift card di 10.00 euro)
            \item $b_{FastBus} = 29.70$ euro
        \end{itemize}
        \item con Dati\_PdTo\_1.dat
        \begin{itemize}
            \item $b_{FerrovieItaliane} = 176.10$ euro (usando la gift card di 10.00 euro)
            \item $b_{BusTravel} = 3.99$ euro (usando la gift card di 10.00 euro)
            \item $b_{FastBus} = 19.80$ euro
        \end{itemize}
    \end{itemize}
\end{itemize}
Si può osservare come, richiedendo di fare al massimo tre tratte con ogni compagnia, è stato necessario soddisfare all'uguaglianza 
il vincolo sul massimo numero di tratte per poter al contempo soddisfare quello sul numero di viaggi. Rimuovendo il limite delle tre tratte per compagnia,
lasciando di fatto libera scelta (come spiegato in §4.1), il modello ha cercato di puntare sull'offerta di FerrovieItaliane che offre certe tratte a prezzi 
tutto sommato contenuti (pur impiegandoci poco tempo), mentre per colmare le rimanenti tratte ha puntato sulle due compagnie di autotrasporti BusTravel e FastBus, 
che però offrono tempi maggiori di percorrenza. In entrambi i casi il budget è consumato quasi del tutto, ma è presente una notevole differenza fra il valore finale 
della funzione obiettivo nel primo e nel secondo file: nel secondo caso infatti si impiegano ben 4.20h in meno, corrispondenti quasi al tempo medio di una tratta in 
questo tragitto.

\subsection*{Tragitto: Milano $\leftrightarrow$ Roma}
Viene richiesta una pianificazione con questi dati:
\begin{itemize}
    \item Budget (massimo): \e{400};
    \item Durata (massima di una tratta): 8.5h;
    \item Numero viaggi da pianificare: 8;
    \item con Dati\_MiRm\_0.dat: $maxTratte^T = [16,16,4,2]$;
    \item con Dati\_MiRm\_1.dat: $maxTratte^T = [16,16,16,16]$.
\end{itemize}
Il risultato ottenuto è il seguente:
\begin{itemize}
    \item totale di ore in viaggio minime per la pianificazione richiesta:
    \begin{itemize}
        \item con Dati\_MiRm\_0.dat: 92.10h
        \item con Dati\_MiRm\_1.dat: 91.85h
    \end{itemize}
    \item corse selezionate (mostrate solo quelle con $x_{t,c,v} > 0$):
    \begin{itemize}
        \item con Dati\_MiRm\_0.dat
        \begin{itemize}
            \item $x_{Trenia,6,Andata} = 5$
            \item $x_{FerrovieItaliane,7,Ritorno} = 8$
            \item $x_{BusTravel,1,Andata} = 1$
            \item $x_{FastBus,1,Andata} = 2$
        \end{itemize}
        \item con Dati\_MiRm\_1.dat
        \begin{itemize}
            \item $x_{Trenia,6,Andata} = 5$
            \item $x_{FerrovieItaliane,7,Ritorno} = 8$
            \item $x_{FastBus,1,Andata} = 3$
        \end{itemize}
    \end{itemize}
    \item budget impiegato per ogni compagnia (mostrati solo quelli con $b_t > 0$):
    \begin{itemize}
        \item con Dati\_MiRm\_0.dat
        \begin{itemize}
            \item $b_{Trenia} = 199.50$ euro (usando la gift card di 15.00 euro)
            \item $b_{FerrovieItaliane} = 149.20$ euro (usando la gift card di 10.00 euro)
            \item $b_{BusTravel} =  8.99$ euro (usando la gift card di 10.00 euro)
            \item $b_{FastBus} = 33.80$ euro
        \end{itemize}
        \item con Dati\_MiRm\_1.dat
        \begin{itemize}
            \item $b_{Trenia} = 199.50$ euro (usando la gift card di 15.00 euro)
            \item $b_{FerrovieItaliane} = 149.20$ euro (usando la gift card di 10.00 euro)
            \item $b_{FastBus} = 50.70$ euro
        \end{itemize}
    \end{itemize}
\end{itemize}
In questo caso invece, con le due istanze fornite i risultati dell'esecuzione del problema non sono granché differenti: i valori delle funzioni obiettivo 
distano di solo 0.25h. Sono stati volontariamente inseriti in questo modo per mostrare come il modello tenda in ogni caso a prediligere le compagnie Trenia e 
FerrovieItaliane dato che in quest'istanza di problema offrono tratte con durata minore. È cambiato invece di quasi \e{8} la differenza di prezzo, ovvero a 
circa \e{8} in più si ottiene una diminuzione del tempo totale di solo 15 minuti (0.25h corrisponde ad un quarto d'ora). Questo può indicare come, aggiustando 
certi parametri (in questo caso $maxTratte$), si possa ottenere una diminuzione del prezzo finale a discapito di un piccolo aumento del tempo di percorrenza, 
che è comunque un altro fattore importante per il principale target del sito web ``Viaggiare LeggeRO": lo studente.

\subsection*{Tragitto: Bologna $\leftrightarrow$ Bolzano}
Viene richiesta una pianificazione con questi dati:
\begin{itemize}
    \item Budget (massimo): \e{100};
    \item Durata (massima di una tratta): 4h;
    \item Numero viaggi da pianificare: 4;
    \item con Dati\_BoBz\_0.dat: $maxTratte^T = [1,1,2,3,1]$;
    \item con Dati\_BoBz\_1.dat: $maxTratte^T = [8,8,8,8,8]$.
\end{itemize}
Il risultato ottenuto è il seguente:
\begin{itemize}
    \item totale di ore in viaggio minime per la pianificazione richiesta:
    \begin{itemize}
        \item con Dati\_BoBz\_0.dat: 25.70h
        \item con Dati\_BoBz\_1.dat: 23.45h
    \end{itemize}
    \item corse selezionate (mostrate solo quelle con $x_{t,c,v} > 0$):
    \begin{itemize}
        \item con Dati\_BoBz\_0.dat
        \begin{itemize}
            \item $x_{Trenia,2,Andata} = 1$
            \item $x_{FerrovieItaliane,5,Andata} = 1$
            \item $x_{DeutscherZug,1,Andata} = 2$
            \item $x_{BusTravel,5,Ritorno} = 3$
            \item $x_{FastBus,2,Ritorno} = 1$
        \end{itemize}
        \item con Dati\_BoBz\_1.dat
        \begin{itemize}
            \item $x_{Trenia,1,Andata} = 1$
            \item $x_{Trenia,2,Andata} = 3$
            \item $x_{Trenia,2,Ritorno} = 1$
            \item $x_{BusTravel,5,Ritorno} = 3$
        \end{itemize}
    \end{itemize}
    \item budget impiegato per ogni compagnia (mostrati solo quelli con $b_t > 0$):
    \begin{itemize}
        \item con Dati\_BoBz\_0.dat
        \begin{itemize}
            \item $b_{Trenia} = 20.90$ euro
            \item $b_{FerrovieItaliane} = 19.90$ euro
            \item $b_{DeutscherZug} = 34.80$ euro (usando la gift card di 5.00 euro)
            \item $b_{BusTravel} = 13.97$ euro (usando la gift card di 10.00 euro)
            \item $b_{FastBus} = 8.90$ euro
        \end{itemize}
        \item con Dati\_BoBz\_1.dat
        \begin{itemize}
            \item $b_{Trenia} = 85.50$ euro (usando la gift card di 15.00 euro)
            \item $b_{BusTravel} = 13.97$ euro (usando la gift card di 10.00 euro)
        \end{itemize}
    \end{itemize}
\end{itemize}
In questa istanza di problema invece si può osservare come le forti limitazioni imposte al numero di tratte in Dati\_BoBz\_0.dat non hanno fatto
cambiare molto la spesa totale da Dati\_BoBz\_1.dat, nonostante la distribuzione della spesa sia invece passata da 5 a 2 compagnie. Ciò che invece è cambiato
molto, seppur possa non sembrare, è il valore della funzione obiettivo: rimuovendo le limitazioni infatti le ore viaggiate diminuiscono di 2.25 che, un po' come 
nel tragitto Padova $\leftrightarrow$ Torino, corrispondono a uno dei tempi minori fra le tratte a disposizione.

\end{document}