\documentclass[main.tex]{subfiles}
\begin{document}

\section{Esecuzione del problema}
\subsection{File dati}
Per provare il modello vengono forniti sei file \textbf{.dat}, ognuno contenente i dati a disposizione del sito al momento della richiesta, oltre alle richieste
dell'utente indicate in §1.1:
\begin{itemize}
    \item Dati\_PdTo\_0.dat, per il tragitto Padova $\leftrightarrow$ Torino (corrispondente ai dati presentati in §1.2);
    \item Dati\_PdTo\_1.dat, per il tragitto Padova $\leftrightarrow$ Torino;
    \item Dati\_MiRm\_0.dat, per il tragitto Milano $\leftrightarrow$ Roma;
    \item Dati\_MiRm\_1.dat, per il tragitto Milano $\leftrightarrow$ Roma;
    \item Dati\_BoBz\_0.dat, per il tragitto Bologna $\leftrightarrow$ Bolzano;
    \item Dati\_BoBz\_1.dat, per il tragitto Bologna $\leftrightarrow$ Bolzano.
\end{itemize}
Per il tragitto Bologna $\leftrightarrow$ Bolzano, il sito web ha a disposizione i dati della compagnia oltre a quelli delle compagnie già presentate. 
Questa compagnia offre un buono sconto di \e{5} fin dal primo acquisto, come la compagnia BusTravel.
Più precisamente, le coppie di file:
\begin{itemize}
    \item Dati\_PdTo\_0.dat e Dati\_PdTo\_1.dat,
    \item Dati\_MiRm\_0.dat e Dati\_MiRm\_1.dat,
    \item Dati\_BoBz\_0.dat e Dati\_BoBz\_1.dat,
\end{itemize}
presentano gli stessi dati, ad eccezione del parametro $maxTratte \in \mathbb{Z}_+^{|T|}$ che nel secondo di ciascuna coppia di file viene istanziato 
con dati che lo rendono di fatto ridondante (poiché viene istanziato con il massimo numero di tratte che deve essere necessariamente effettuato).
Il motivo di questa scelta è dovuto al fatto di voler analizzare quanto varia la soluzione del modello rendendo meno stringenti i vincoli.

\subsection{Risultati dell'esecuzione}
\subsection*{Tragitto: Padova $\leftrightarrow$ Torino}
Viene richiesta una pianificazione con questi dati:
\begin{itemize}
    \item Budget (massimo): \e{200};
    \item Durata (massima di una tratta): 5h;
    \item Numero viaggi da pianificare: 6;
    \item Dati\_PdTo\_0.dat: $maxTratte^T = [3,3,3,3]$;
    \item Dati\_PdTo\_1.dat: $maxTratte^T = [12,12,12,12]$.
\end{itemize}
Il risultato ottenuto è il seguente:
\begin{itemize}
    \item valore della funzione obiettivo:
    \begin{itemize}
        \item Dati\_PdTo\_0.dat: 49.20h
        \item Dati\_PdTo\_1.dat: 45.00h
    \end{itemize}
    \item corse selezionate (con $x_{t,c,v} > 0$):
    \begin{itemize}
        \item Dati\_PdTo\_0.dat
        \begin{itemize}
            \item $x_{Trenia,4,Ritorno} = 3$
            \item $x_{FerrovieItaliane,1,Andata} = 3$
            \item $x_{BusTravel,3,Andata} = 3$
            \item $x_{FastBus,3,Ritorno} = 3$
        \end{itemize}
        \item Dati\_PdTo\_1.dat
        \begin{itemize}
            \item $x_{Trenia,4,Ritorno} = 5$
            \item $x_{FerrovieItaliane,1,Andata} = 1$
            \item $x_{FerrovieItaliane,7,Andata} = 4$
            \item $x_{FerrovieItaliane,3,Ritorno} = 1$
            \item $x_{BusTravel,1,Andata} = 1$
        \end{itemize}
    \end{itemize}
    \item budget impiegato per ogni compagnia (con $b_t > 0$):
    \begin{itemize}
        \item Dati\_PdTo\_0.dat
        \begin{itemize}
            \item $b_{Trenia} = 53.70$ euro
            \item $b_{FerrovieItaliane} = 80.70$ euro
            \item $b_{BusTravel} = 31.97$ euro (usando la gift card di 10.00 euro)
            \item $b_{FastBus} = 29.70$ euro
        \end{itemize}
        \item Dati\_PdTo\_1.dat
        \begin{itemize}
            \item $b_{Trenia} = 74.50$ euro (usando la gift card di 15.00 euro)
            \item $b_{FerrovieItaliane} = 123.40$ euro (usando la gift card di 10.00 euro)
            \item $b_{BusTravel} = 1.99$ euro (usando la gift card di 10.00 euro)
        \end{itemize}
    \end{itemize}
\end{itemize}

\subsection*{Tragitto: Milano $\leftrightarrow$ Roma}
Viene richiesta una pianificazione con questi dati:
\begin{itemize}
    \item Budget (massimo): \e{400};
    \item Durata (massima di una tratta): 8.5h;
    \item Numero viaggi da pianificare: 8;
    \item Dati\_MiRm\_0.dat: $maxTratte^T = [16,16,4,2]$;
    \item Dati\_MiRm\_1.dat: $maxTratte^T = [16,16,16,16]$.
\end{itemize}
Il risultato ottenuto è il seguente:
\begin{itemize}
    \item valore della funzione obiettivo:
    \begin{itemize}
        \item Dati\_MiRm\_0.dat: 92.10h
        \item Dati\_MiRm\_1.dat: 91.85h
    \end{itemize}
    \item corse selezionate (con $x_{t,c,v} > 0$):
    \begin{itemize}
        \item Dati\_MiRm\_0.dat
        \begin{itemize}
            \item $x_{Trenia,6,Andata} = 5$
            \item $x_{FerrovieItaliane,7,Ritorno} = 8$
            \item $x_{BusTravel,1,Andata} = 1$
            \item $x_{FastBus,1,Andata} = 2$
        \end{itemize}
        \item Dati\_MiRm\_1.dat
        \begin{itemize}
            \item $x_{Trenia,6,Andata} = 5$
            \item $x_{FerrovieItaliane,7,Ritorno} = 8$
            \item $x_{FastBus,1,Andata} = 3$
        \end{itemize}
    \end{itemize}
    \item budget impiegato per ogni compagnia (con $b_t > 0$):
    \begin{itemize}
        \item Dati\_MiRm\_0.dat
        \begin{itemize}
            \item $b_{Trenia} = 199.50$ euro (usando la gift card di 15.00 euro)
            \item $b_{FerrovieItaliane} = 149.20$ euro (usando la gift card di 10.00 euro)
            \item $b_{BusTravel} =  8.99$ euro (usando la gift card di 10.00 euro)
            \item $b_{FastBus} = 33.80$ euro
        \end{itemize}
        \item Dati\_MiRm\_1.dat
        \begin{itemize}
            \item $b_{Trenia} = 199.50$ euro (usando la gift card di 15.00 euro)
            \item $b_{FerrovieItaliane} = 149.20$ euro (usando la gift card di 10.00 euro)
            \item $b_{FastBus} = 50.70$ euro
        \end{itemize}
    \end{itemize}
\end{itemize}

\subsection*{Tragitto: Bologna $\leftrightarrow$ Bolzano}
Viene richiesta una pianificazione con questi dati:
\begin{itemize}
    \item Budget (massimo): \e{100};
    \item Durata (massima di una tratta): 4h;
    \item Numero viaggi da pianificare: 4;
    \item Dati\_BoBz\_0.dat: $maxTratte^T = [1,1,2,3,1]$;
    \item Dati\_BoBz\_1.dat: $maxTratte^T = [8,8,8,8,8]$.
\end{itemize}
Il risultato ottenuto è il seguente:
\begin{itemize}
    \item valore della funzione obiettivo:
    \begin{itemize}
        \item Dati\_BoBz\_0.dat: 25.70h
        \item Dati\_BoBz\_1.dat: 23.45h
    \end{itemize}
    \item corse selezionate (con $x_{t,c,v} > 0$):
    \begin{itemize}
        \item Dati\_BoBz\_0.dat
        \begin{itemize}
            \item $x_{Trenia,2,Andata} = 1$
            \item $x_{FerrovieItaliane,5,Andata} = 1$
            \item $x_{DeutscherZug,1,Andata} = 2$
            \item $x_{BusTravel,5,Ritorno} = 3$
            \item $x_{FastBus,2,Ritorno} = 1$
        \end{itemize}
        \item Dati\_BoBz\_1.dat
        \begin{itemize}
            \item $x_{Trenia,1,Andata} = 1$
            \item $x_{Trenia,2,Andata} = 3$
            \item $x_{Trenia,2,Ritorno} = 1$
            \item $x_{BusTravel,5,Ritorno} = 3$
        \end{itemize}
    \end{itemize}
    \item budget impiegato per ogni compagnia (con $b_t > 0$):
    \begin{itemize}
        \item Dati\_BoBz\_0.dat
        \begin{itemize}
            \item $b_{Trenia} = 20.90$ euro
            \item $b_{FerrovieItaliane} = 19.90$ euro
            \item $b_{DeutscherZug} = 34.80$ euro (usando la gift card di 5.00 euro)
            \item $b_{BusTravel} = 13.97$ euro (usando la gift card di 10.00 euro)
            \item $b_{FastBus} = 8.90$ euro
        \end{itemize}
        \item Dati\_BoBz\_1.dat
        \begin{itemize}
            \item $b_{Trenia} = 85.50$ euro (usando la gift card di 15.00 euro)
            \item $b_{BusTravel} = 13.97$ euro (usando la gift card di 10.00 euro)
        \end{itemize}
    \end{itemize}
\end{itemize}

\end{document}