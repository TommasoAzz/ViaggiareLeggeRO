\documentclass[main.tex]{subfiles}
\begin{document}

\section{Modello matematico}
In questa sezione viene illustrato il modello matematico di Programmazione Lineare utilizzato per la risoluzione del problema descritto nella precedente sezione.\\
Questo modello verrà successivamente descritto nel linguaggio di programmazione \textbf{AMPL}.

\subsection*{Insiemi}
Per questo problema vengono individuati tre insiemi:
\begin{itemize}
    \item $T$, l'insieme delle compagnie di trasporto;
    \item $C$, l'insieme delle possibili corse di una compagnia. Corrisponde alla cardinalità massima fra tutte le cardinalità delle corse disponibili per una compagnia;
    \item $V$, l'insieme dei versi di una tratta. È fisso: $V = \{Andata, Ritorno \}$.
\end{itemize}
Per i dati di esempio della sezione §2.2:
\begin{itemize}
    \item $T = \{Trenia, FerrovieItaliane, BusTravel, FastBus\}$;
    \item $C = \{1, 2, 3, 4, 5, 6, 7\}$.
\end{itemize}

\subsection*{Variabili}
Le variabili individuate sono le seguenti (in ordine alfabetico):
\begin{itemize}
    \item $b_t \in \mathbb{R}_+$, corrispondente all'ammontare del budget totale impiegato per la compagnia $t \in T$
    \item $w =$ 
    \begin{math} { \begin{cases}
        1, & necessario\; pi\grave{u}\; tempo\; aggiuntivo\; per\; almeno\; una\; corsa\; per\; rientrare\; nel\; budget \\
        0, & non\; necessario\; pi\grave{u}\; tempo\; per\; almeno\; una\; corsa\; per\; rientrare\; nel\; budget
    \end{cases} } \end{math}
    \item $x_{t,c,v} \in \mathbb{Z}_+$, corrispondente al numero di volte in cui è stata scelta la corsa $c \in C$ della compagnia $t \in T$ nel verso $v \in V$
    \item $y_{t,c,v} =$
    \begin{math} { \begin{cases}
        1, & la\; corsa\; \text{$c \in C$}\; della\; compagnia\; \text{$t \in T$}\; nel\; verso\; \text{$v \in V$}\; pu\grave{o}\; essere\; scelta \\
        0, & la\; corsa\; \text{$c \in C$}\; della\; compagnia\; \text{$t \in T$}\; nel\; verso\; \text{$v \in V$}\; non\; pu\grave{o}\; essere\; scelta
    \end{cases} } \end{math}
    \item $z_t =$
    \begin{math} { \begin{cases}
        1, & la\; gift\; card\; della\; compagnia\; \text{$t \in T$}\; viene\; emessa \\
        0, & la\; gift\; card\; della\; compagnia\; \text{$t \in T$}\; non\; viene\; emessa 
    \end{cases} } \end{math}
\end{itemize}
\textbf{N.B.} $0 \in \mathbb{R}_+, \mathbb{Z}_+$.

\subsection*{Parametri}
I parametri richiesti per poter risolvere il problema sono i seguenti:
\begin{itemize}
    \item $p_{t,c,v} \in \mathbb{R}_+$, corrispondente al prezzo della corsa (in euro) $c \in C$ nel verso $v \in V$ con la compagnia $t \in T$;
    \item $d_{t,c,v} \in \mathbb{R}_+$, corrispondente alla durata della corsa (in ore) $c \in C$ nel verso $v \in V$ con la compagnia $t \in T$ (essendo in ore, due ore e mezza (2h30), ad esempio, si indicano come 2.5 ore);
    \item $B \in \mathbb{R}_+$, corrispondente al budget massimo (in euro) per tutti i viaggi da pianificare;
    \item $D \in \mathbb{R}_+$, corrispondente alla durata massima (in ore) di una tratta;
    \item $N \in \mathbb{Z}_+$, corrispondente al numero di viaggi da pianificare;
    \item $nViaggiGC_t \in \mathbb{Z}_+$, corrispondente al numero di viaggi necessari per l'emissione della gift card da parte della compagnia $t \in T$;
    \item $vScontoGC_t \in \mathbb{R}_+$, corrispondente al valore della gift card emessa da parte della compagna $t \in T$;
    \item $tempoAgg \in \mathbb{R}_+$, corrispondente al tempo aggiuntivo richiesto per almeno un viaggio pur di rientrare nel budget;
    \item $M \in \mathbb{R}_+$, corrispondente a un numero molto grande, necessario per l'attivazione delle variabili binarie.
\end{itemize}

\subsection*{Modello}
Funzione obiettivo
$$min \sum_{t\; \in\; T} \sum_{c\; \in\; C} \sum_{v\; \in\; V} d_{t,c,v} \times{} x_{t,c,v}$$
Il budget totale a disposizione viene diviso per tutte le compagnie (è necessario per applicare correttamente gli sconti dati dalle gift card)
$$\sum_{t\; \in\; T} b_t \leq B$$
Va rispettato il budget complessivo per tutti i viaggi acquistati, tenendo conto della possibilità di usare le gift card
$$\sum_{c\; \in\; C} \sum_{v\; \in\; V} p_{t,c,v} \times{} x_{t,c,v} = b_t + vScontoGC_t \times{} z_t \hspace{10mm} \forall{} t \in T$$
Possono essere scelte solo le corse che rispettano il tempo massimo per corsa richiesto (eventualmente considerando la possibilità di aggiungere del tempo)
$$d_{t,c,v} \times{} y_{t,c,v} \leq D + tempoAgg \times w \hspace{10mm} \forall{} t \in T, \forall{} c \in C, \forall{} v \in V$$
Devono essere pianificati esattamente tanti viaggi quanti richiesti
$$\sum_{t\; \in\; T} \sum_{c\; \in\; C} x_{t,c,v} = N \hspace{10mm} \forall{} v \in V$$
Attivazione delle variabili binarie (creazione di un legame fra $x$ e $y$, infatti $\nexists y_{t,c,v}, \nexists x_{t,c,v}\; t.c.\; y_{t,c,v} = 0, x_{t,c,v} > 0\; \forall{} t \in T, \forall{} c \in C, \forall{} v \in V$)
$$x_{t,c,v} \leq y_{t,c,v} \times{} M \hspace{10mm} \forall{} t \in T, \forall{} c \in C, \forall{} v \in V$$
Emissione delle gift card
$$\sum_{c\; \in\; C} \sum_{v\; \in\; V} x_{t,c,v} \geq nViaggiGC_t \times z_t \hspace{10mm} \forall{} t \in T$$
\end{document}